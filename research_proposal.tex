\documentclass[11pt]{article}
\usepackage{amsmath}
\usepackage{amssymb}
\usepackage{hyperref}
\usepackage{graphicx}
\usepackage{color}
\usepackage{cite}
\usepackage{framed}
\usepackage[margin=1in]{geometry}
\usepackage[parfill]{parskip}
\setlength{\parindent}{0pt}
\begin{document}

\title{Charge and Lattice Dynamics in Quantum Matter}
\author{Xunyang Hong}
\date{}
\maketitle





\begin{abstract}
This proposal presents a focused study on electron-phonon coupling in quantum matter using resonant inelastic X-ray scattering (RIXS) across three projects.

The first project aims to investigate the interaction between charge order and electron-phonon coupling in cuprate superconductors, particularly $\mathrm{La_{1.8-x}Eu_{0.2}Sr_xCuO_{4}}$, to understand interplay between phonons and charge order excitations.

The second project explores the electron-phonon dynamics in $\mathrm{SrTiO_{3}}$ and $\mathrm{KTaO_{3}}$, examining the reversed isotope effect and orientation-dependent superconductivity, respectively, by measurement of the electron-phonon coupling via RIXS.

Finally, the study extends to sapphire to measure its electron-phonon coupling, aiming to facilitate sub-MeV dark matter detection.

These projects collectively aim to deepen our understanding of electron-phonon interactions and their implications in superconductivity and dark matter research.
\end{abstract}

\section{Objectives of the project}
% \textit{\textbf{Instruction}: Please present the rationale for your project based on the current state of knowledge in the respective field, list the {\color{cyan}general research question} and the {\color{cyan}specific objectives}, mention the {\color{cyan}research methods}, and briefly discuss the {\color{cyan}expected results} and their implications for your field}

\paragraph{Research question and objectives}
Electron-phonon coupling is fundamental to the understanding of many phenomena in condensed matter physics, such as superconductivity\cite{bardeen_theory_1957,cuk_review_2005}, charge order\cite{arpaia_charge_2021,comin_resonant_2016,tranquada_spins_2013}, and even dark matter detection\cite{griffin_directional_2018}. 

In conventional superconductors that obeys BCS theory, phonon is believed to mediate  the superconductivity\cite{bardeen_theory_1957}, and thus the electron-phonon coupling is crucial to the understanding of conventional superconductivity. In high-temperature superconductors, however, even if the underlying mechanism is not fully understood, the role of electron-phonon coupling in driving superconductivity is still very interesting to study.

In the course of my PhD. research, I propose 3 projects to study electron-phonon coupling in different systems, using the state-of-the-art high-resolution resonant inelastic x-ray scattering (RIXS) technique\cite{ament_resonant_2011,zhou_i21_2022}. 

\paragraph{Project 1: Charge order and electron-phonon coupling in cuprate superconductor}
In cuprate superconductors, the coexistence and interaction of charge order with superconductivity phases are yet to be fully deciphered. The charge order phase is believed to compete against the superconductivity phase\cite{arpaia_charge_2021,comin_resonant_2016,canosa_resonant_2014, hucker_competing_2014, chang_direct_2012,ghiringhelli_long-range_2012}, but the underlying mechanism is not yet fully understood. Recently study suggested that the charge order in cuprate may also interact with phonons: (1) Electron-phonon coupling seems to reinforce the charge order in cuprate and result in a "lock-in" effect \cite{wang_charge_2021}; (2) phonon softening and intensity anomaly were observed near the charge order wavevector\cite{wang_charge_2021,lin_strongly_2020, huang_quantum_2021,miao_incommensurate_2018,tacon_inelastic_2014,li_multiorbital_2020,braicovich_determining_2020,chaix_dispersive_2017,peng_enhanced_2020};  However, On the other hand, a recent study, which I was also involved, done in the group at University of Zurich led by Prof. Johan Chang suggested the \textit{opposite} : phonons and charge order excitations are decoupled in LSCO.

This observed elusive relation among phonons, charge orders, and superconductivity underscores the significance of electron-phonon coupling. On one hand, electron-phonon coupling (especially its momentum dependence) determins how phonon can theoretically interact with charge order; on the other hand, Both charge order and electron-phonon coupling plays a role in shaping the behavior of superconductivity. Therefore, we propose to explore the electron-phonon coupling and how phonon interacts with charge order in a $\mathrm{La_{1.8-x}Eu_{0.2}Sr_xCuO_{4}}$ in our project. We expect to unravel the mysterious relation between phonon and charge order in cuprate superconductors. 

\paragraph{Project 2: Electron-phonon coupling in $\mathrm{SrTiO_{3}}$ and $\mathrm{KTaO_{3}}$} 
$\mathrm{SrTiO_{3}}$ is a well-known superconducting material with a critical temperature of $\lesssim 1\, \mathrm{K}$\cite{schooley_superconductivity_1964,lin_fermi_2013}. Although it has a relatively simple crystal structure, the mechanism of superconductivity in $\mathrm{SrTiO_{3}}$ is still a mystery. Among all the puzzles, the most intriguing one is the reversed isotope effect: by substutiting ${}^{16}\mathrm{O}$ with ${}^{18}\mathrm{O}$, the critical temperature increases, which contradicts the conventional BCS theory\cite{stucky_isotope_2016}. We propose to investigate this effect by studying the electron-phonon coupling in $\mathrm{SrTiO_{3}}$ using RIXS. If the reversed istope effect is also seen in electron-phonon coupling, it would be a strong evidence that electron-phonon coupling plays a role in $\mathrm{SrTiO_{3}}$ superconductivity.

Another material, $\mathrm{KTaO_{3}}$, which is not superconducting in bulk, is also of great interest when superconductivity was observed on $\mathrm{LaAlO_{3}/KTaO_{3}}$ (LAO/KTO) interface recently\cite{ren_two-dimensional_2022}. However, unlike LAO/$\mathrm{SrTiO_{3}}$ interface where the superconductivity was found in all directions, the LAO/KTO interface exhibits a strong orientation-dependence of critical temperature\cite{ren_two-dimensional_2022,chen_two-dimensional_2021}. Similar to the hypothesis in STO, The orientation-dependent superconductivity observed in LAO/KTO may be attributed to the anisotropic electron-phonon coupling. To gain a deeper understanding of this phenomenon, it is necessary to quantitively and experimentally measure the electron-phonon coupling in LAO/KTO in different direction.


Therefore, we propose to study the isotope effect of electron-phonon coupling in $\mathrm{SrTiO_{3}}$, and the orientation dependence of electron-phonon coupling in LAO/KTO interface. Resonant inelastic X-ray scattering (RIXS) will be our main research tool. We expect to see a similar trend in electron-phonon coupling as in the superconducting critical temperature $T_{c}$, if electron-phonon coupling does play a role in superconductivity in these materials.



\paragraph{Project 3: Electron-phonon coupling in sapphire, a potential material for sub-MeV dark matter detection}
The concept of dark matter has long been shrouded in mystery in the field of modern physics. Despite numerous efforts to study and to detect dark matter, the exact mass of the dark matter particle remains unknown. Different experimental tools have been utilized to cover a range of masses for dark matter detection, including phonon-based technique, which can be used to detect the sub-MeV dark matter (10 keV - 1 MeV).
  
In the dark matter model, dark matter particles interact with phonons, in a similar manner to electrons. As such, it would be useful to experimentally measure the electron-phonon coupling in the material being used for dark matter detection. A recent study suggests that sapphire is a suitable material for phonon-based dark matter detection, due to its polar nature, anisotropy, and minimal screening effect.
  
Therefore, we propose to conduct a resonant inelastic x-ray scattering (RIXS) study on sapphire to experimentally extract the electron-phonon coupling. By accurately measuring this coupling, we expect to facilitate the detection of dark matter on sapphire.

All of the above projects are based on the measurement of electron-phonon coupling. This requires the use of RIXS.

\paragraph{Research method: resonant inelastic X-ray scattering (RIXS)}
Extracting electron-phonon coupling has long been a difficult problem in condensed matter physics. Before the advent of RIXS, conventional experimental methods, such as neutron scattering or Raman scattering, are not directly related to the electron-phonon coupling. With the help of RIXS, the electron-phonon coupling can be extracted in a more direct way. The idea to probe electron-phonon coupling with RIXS was first proposed by Ament et al. in 2011 \cite{ament_resonant_2011}. In their paper, they proposed a model to describe the phonon intensity in RIXS, with reasonable simplification and approximation. Later on, more sophisticated models were proposed to describe the phonon intensity in RIXS using Feynman diagrams \cite{devereaux_directly_2016,matsubayashi_numerical_2023} or density matrix renormalization group method\cite{nocera_computing_2018}, which allows us to \textit{simulate}  the RIXS spectrum, especially the spectrum with phonons. 

According to the theoretical model, phonon intensity in RIXS is \textit{affected}  by the electron-phonon coupling\cite{devereaux_directly_2016}. In the case where electron-phonon coupling is independent of the momentum of the excited electron, the phonon intensity in the RIXS spectrum is \textit{directly proportional}  to the square of the electron-phonon coupling. With further approximation where electron-phonon coupling is a pure constant, the \textit{relative} value of the elctron-phonon coupling can thus be calculated directly from the phonon intensity in the RIXS spectrum. Moreover, the \textit{abosolute} value of electron-phonon coupling can be further extracted in two ways: (1) by comparing the phonon intensity and its overtones; or (2) by detuning the incident energy and comparing the change in phonon intensity\cite{ament_resonant_2011,braicovich_determining_2020}. Further development in RIXS theory propose modification in Ament's model to account for the momentum dependence of electron-phonon coupling, relaxing some of the strict assumptions in Ament's model\cite{bieniasz_theory_2022,geondzhian_generalization_2020}. 

It turns out that RIXS is a perfect tool for our projects, due to its strong power in measuring phonons and electron-phonon coupling. Thanks to the recent development in RIXS, the resolution at $\mathrm{Cu}$ edge and $\mathrm{O}$ edge can reach up to $\sim 50\,\mathrm{meV}$ and $\sim 30\,\mathrm{meV}$ respectivly, making it possible to seperate phonon signal and pure elastic signal. Here we briefly summarize the use of RIXS in the three projects:
\begin{itemize}
  \item In project 1 (cuprate superconductor project), both charge order and phonons can be observed by RIXS. By simulating phonon and charge order spectrum seperately, we can study the interplay between phonons and charge order. 
  \item In project 2 (STO and KTO project) and project 3 (sapphire project), the use of RIXS is more straightforward: accquiring the RIXS spectrum and directly extracting the electron-phonon coupling using Ament's model.
\end{itemize}

\section{Background Information and Current State of Research}
% \textit{\textbf{Instructions}: Describe your project in the context of the {\color{cyan}current state of research} in your field. Refer to the most important publications, especially by other authors. {\color{cyan}Describe which previous findings are the starting point and basis for the planned studies}, where and why there is a need for research and which important, relevant research is currently underway in Switzerland and abroad. } 

The first superconductor was discovered in Hg in 1911 by Kamerlingh Onnes. Since then, superconductivity has been observed in many materials, and the mechanism of superconductivity has been studied extensively. In 1957, Bardeen, Cooper, and Schrieffer proposed the BCS theory, which is the first successful theory to explain superconductivity\cite{bardeen_theory_1957}. In BCS theory, phonon is believed to mediate the superconductivity. More specifically, interaction between phonons and electrons creates an effective attractive interation between electrons, creating a so-called cooper pair that leads to superconductivity at low temperature\cite{bardeen_theory_1957}. 

In 1986, Bednorz and Müller discovered a $\mathrm{La}$-based cuprate material that becomes superconducting at a striking $35\,\mathrm{K}$\cite{bednorz_possible_1986}, heralding a new era of high-temperature superconductor. Since then, many  high-temperature superconductors have been discovered, including $\mathrm{YBa_{2}Cu_{3}O_{7-x}}$\cite{wu_superconductivity_1987}, $\mathrm{BiSrCaCu_{2}O_{x}}$\cite{maeda_a_1988}, and more. However, this type of superconductor cannot be explained by BCS theory, and the underlying mechanism is still under intense debate. Many new theories haven been put forward, and yet, even if BCS theory is not able to explain the mechanism, the concept of phonon-mediated interaction is still believed to play a role in high-temperature superconductivity. There are several reasons: (1) It's theoretically predicted that phonon together with other pairing-driving excitations could enhance superconducting critical temperature\cite{braicovich_determining_2020}; (2) apical oxygen phonon mode in some cuprate materials seems to promote supercondutivity in cuprate under certain optical inducement\cite{kaiser_optically_2014}; and (3) charge order in cuprate is assumed to compete with superconducting phase, while phonon was observed to interact with both static and dynamical charge order as well\cite{arpaia_charge_2021,comin_resonant_2016,canosa_resonant_2014, hucker_competing_2014, chang_direct_2012,ghiringhelli_long-range_2012,wang_charge_2021,lin_strongly_2020, huang_quantum_2021,miao_incommensurate_2018,tacon_inelastic_2014,li_multiorbital_2020,braicovich_determining_2020,chaix_dispersive_2017,peng_enhanced_2020}. Therefore, a better understanding of phonon, especially its interaction with electron and electronic orders, will certainly shed light on the mechanism of superconductivity.  

My first project is proprosed based on the third point mentioned above: the interplay between phonon and charge order. Charge order is present in many families in cuprate superconductors and shows clear interplay with superconducting phase. Chagre density wave in La-based cuprate is believed to stem from a strong electronic interactions which results in a so-called stripe order\cite{harriger_stripe_nodate,tranquada_evidence_1995,choi_disentangling_2020}. The stripe order is different from the usual charge order induced by special features in momentum-dependet electron-phonon coupling. However, phonon anomalies observed in the vicinity to the stripe order indicates a \textit{possible} interaction between phonon and both static and dynamical charge order: 
\begin{itemize}
\item \textit{Phonon softening}  were observed near the charge order wavevector In $\mathrm{La_{2-x}Sr_{x}CuO_{4}}$\cite{lin_strongly_2020, wang_charge_2021, huang_quantum_2021}, $\mathrm{La_{2-x}Ba_{x}CuO_{4}}$\cite{miao_incommensurate_2018}, $\mathrm{YBa_{2}Cu_{3}O_{6+x}}$\cite{tacon_inelastic_2014}, $\mathrm{Bi_{2}Sr_{2}LaCuO_{6+\delta}}$\cite{li_multiorbital_2020}, and ect. It's still not clear what leads to this phenomenon, as many factors could contribute to phonon softening, such as pure lattice distortion\cite{lin_strongly_2020}, and interaction between phonon and other excitations.  
\item \textit{Enhancement in phonon intensity}  in RIXS measurement was observed in $\mathrm{Bi_{2}Sr_{2}LaCuO_{6+\delta}}$\cite{li_multiorbital_2020}, $\mathrm{Bi_{2}Sr_{2}CaCu_{2}O_{8+\delta}}$\cite{chaix_dispersive_2017}, $\mathrm{Nd_{1+x}Ba_{2-x}Cu_{3}O_{7+\delta}}$\cite{braicovich_determining_2020}, and $\mathrm{La_{1.8-x}Eu_{0.2}Sr_xCuO_{4}}$\cite{peng_enhanced_2020,wang_charge_2021,huang_quantum_2021}. However, it was also reported that phonon intensity in $\mathrm{La_{2-x}Sr_{x}CuO_{4}}$ remains relatively unchanged across a wide range of doping, even when charge order disappears\cite{lin_strongly_2020}.  
\end{itemize}

Interestingly, a recent study (paper in pipeline) done in a quantum matter research group led by Johan Chang at University of Zurich applied uniaxial pressure to engineer the intensity of charge order in $\mathrm{La_{2-x}Sr_{x}CuO_{4}}$, and found that phonon intensities do not change even when the charge order intensity is supressed or enhanced. This result suggests that phonon and charge order excitations might be decoupled in $\mathrm{La_{2-x}Sr_{x}CuO_{4}}$, supporting the theory given in Ref.\cite{lin_strongly_2020}, but contradicting some other studies\cite{li_multiorbital_2020, chaix_dispersive_2017,huang_quantum_2021}. 

Due to the nature of RIXS, an enhancement in phonon intensity could stem from different factors: 
\begin{enumerate}
  \item an increase in electron-phonon coupling into the charge order phase as proposed in Ref.\cite{wang_charge_2021,peng_electronic_2022}; or
  \item interaction between phonon and charge order excitations as discussed in Ref.\cite{li_multiorbital_2020, chaix_dispersive_2017,huang_quantum_2021}; or 
  \item \textit{extra excitations overlap with phonon, creating extra RIXS signal on top of phonon spectrum, which has long been thought of as part of phonon itself.} 
\end{enumerate}
\textit{It's worth mentioning that (3) is exactly the starting point of my first project: we hypothesize that the enhancement of phonon intensity in cuprate originates from the charge excitations signal that overlap with phonon signal, instead of interacts with phonon iteself.}  We seek to explore the nature of charge order excitations in $\mathrm{La_{1.8-x}Eu_{0.2}Sr_xCuO_{4}}$ with RIXS under this hypothesis, and furthermore extend our model to other cuprate materials. Our study will provide a new perspective on the relation between phonon and charge order excitations in cuprate superconductors. 

To do this, we will need to use RIXS to perform measurements on $\mathrm{La_{1.8-x}Eu_{0.2}Sr_xCuO_{4}}$, and later on simulate the RIXS spectrum using the model proposed in Ref.\cite{devereaux_directly_2016}. Comparing simulation and experimental results will allow us to study the properties of charge order excitations in this material. 

In addition to cuprate, phonon also plays an important role in some other unconventional superconductors. For example, $\mathrm{SrTiO_{3}}$ can host superconductivity even when the number of charge carriers is as low as $\sim 10^{17}\,\mathrm{cm^{-3}}$\cite{schooley_superconductivity_1964,lin_fermi_2013}, similar to some other dilute systems such as $\mathrm{Pb_{1-x}Tl_{x}Te}$\cite{}, and single crystal $\mathrm{Bi}$\cite{}. It was theoretically suggested that phonon-mediated interaction is a key driving force of superconductivity, leading to an increase of transition temperature in $\mathrm{SrTiO_{3}}$ as the electron density decreases\cite{gastiasoro_phonon-mediated_2019}, i.e. as the system becomes more electronic dilute. Another study on $\mathrm{SrTiO_{3}}$ has observed an polaronic behavior where electronic energy band has renormalized strongly by the longitudinal optic phonon, indicating a strong electron-phonon coupling in this material\cite{swartz_polaronic_2018}. 

However, phonon-mediated interaction alone cannot explain the unconventional superconducting behavior in $\mathrm{SrTiO_{3}}$:
\begin{itemize}
  \item despite a strong interaction between longitudinal optical phonons and electrons, the relatively small superconducting gap indicates a weak effective electron-phonon coupling, different from the conventional superconductor; 
  \item a reverse isotope effect was observed: then superconducting critical temperature increases by $50\%$, when $35\%$ of the ${}^{16}\mathrm{O}$ are replaced by ${}^{18}\mathrm{O}$\cite{stucky_isotope_2016}. This is again contradicting the conventional behavior predicted in BCS theory.
\end{itemize}
Even though, according to the first point, there's a discrepancy between the actual electron-phonon coupling and the required coupling for superconductivity, it's still unknown whether this phonon contribute to inducing superconductivity. We therefore propose to tackle the problem by studying the reverse isotope effect mentioned in the second point. \textit{We plan to perform a RIXS measurement on $\mathrm{SrTiO_{3}}$ with different oxygen substitution ratio, and try to extract electron-phonon coupling as a function of the ratio.}  If this phonon does have  contribution in giving rise to superconductivity, we expect to see a similar trend in electron-phonon coupling as in the superconducting transition temperature $T_{c}$. If this is the case, both questions mentioned above will be much clearer to us. 

Another material, $\mathrm{KTaO_{3}}$ that has similar crystal structure as $\mathrm{SrTiO_{3}}$, has recently been found to host superconductivity on the surface\cite{ren_two-dimensional_2022}. It is interesting to mention that, unlike in $\mathrm{SrTiO_{3}}$, \textit{bulk}  electron-doped $\mathrm{KTaO_{3}}$ is not superconducting, and yet on $\mathrm{LaAlO_{3}/KTaO_{3}}$ interface, superconductivity is possible\cite{ren_two-dimensional_2022,chen_two-dimensional_2021}. Another striking feature worth mentioning is the orientation dependence of superconducting critical temperature $T_{c}$ in the $\mathrm{LaAlO_{3}/KTaO_{3}}$ interface\cite{ren_two-dimensional_2022,chen_two-dimensional_2021}. More specifically: 
\begin{itemize}
\item in $(111)$ direction, $T_{c} = 2\,\mathrm{K}$\cite{ren_two-dimensional_2022};
\item in $(110)$ direction, $T_{c} = 0.9\,\mathrm{K}$\cite{chen_two-dimensional_2021};
\item but in $(001)$ direction, no superconducting phase was found down to $T = 25\,\mathrm{mK}$\cite{ren_two-dimensional_2022} 
\end{itemize}
Similar to $\mathrm{SrTiO_{3}}$\cite{swartz_polaronic_2018}, polaronic behavior has also been observed in $\mathrm{KTaO_{3}}$, again suggesting a strong interaction between electron and longitudinal optical phonon\cite{chen_orientation-dependent_2023}. It is possible that phonon-mediated interaction also plays a part in driving superconductivity in $\mathrm{KTaO_{3}}$. \textit{Therefore, we propose to perform an RIXS measurement on $\mathrm{KTaO_{3}}$ in different direction, and try to extract the electron-phonon coupling.}  We expect to see similar trend in the electron-phonon coupling as in the superconducting critical temperature. 

Both in $\mathrm{SrTiO_{3}}$ and $\mathrm{KTaO_{3}}$, \textit{electron-phonon coupling}  is our main focus. Thanks to the powerful tool, RIXS, it is possible to extract the absolute value of the momentum-dependent electron-phonon coupling directly. Due to the importance of electron-phonon coupling in superconductivity, RIXS serves as a sharp spear pierceing through the hard curst of superconductivity and unraveling the mysterious underlying mechanism. 

Furthermore, the importance of electron-phonon coupling does not only lie in superconductivity, it is also crucial in the field of other cutting-edge research, such as dark matter detection. The detection of dark matter is one of the most important problems in physics, as it is one of the two famous unsolved problems in modern physics, together with dark energy. The concept of dark matter was first proposed by Zwicky in 1933 to account for the extra high gravity in galaxies compared to the actual gravity produced by observable mass. However, the direct detection of dark matter has so far been unsuccessful, due to the weak interaction between dark matter and ordinary matter. Different methods have been put forward to attempt to detect dark matter\cite{}, each of which covers a different range of dark matter mass. Within these method, phonon-based detection is promising in detecting sub-MeV dark matter (10 keV - 1 MeV)\cite{}. Phonon-based detection relies on the interaction between dark matter and phonon, which is similar to the interaction between electron and phonon\cite{}. 

Recently, a new material, sapphire, has been proposed to be a suitable material for phonon-based dark matter detection\cite{}. According to a recent research, sapphire is a suitable material for several reasons. (1) Sapphire is a polar material that is expected to interact strongly with dark matter, facilitated by dark photon mediation. (2) Sapphire is an insulator with minimal screening effects which could potentially hinder the interaction between phonons and dark matter particles; (3)  The anisotropic property of sapphire allows for the differentiation of dark matter particles from other particles. The anisotropy of sapphire means that the electron-phonon coupling as well as dark-matter-phonon coupling displays an anisotropic behavior\cite{}, and thus the cross section for dark matter detection is expected to be also anisotropic. Due to the Earth's rotation, the anisotropic dark matter signal is expected to change with time as the dark matter flux goes through the material in a different direction at different time of a day, depending on the orienation of the Earth. This daily-scaling property enables us to distinguish dark matter signal from other background signals. 

The first step towards detecting sub-MeV dark matter is to measure the electron-phonon coupling in sapphire, as this coupling is assumed to resemble dark-matter-phonon coupling. \textit{Therefore, we propose to perform an RIXS measurement on sapphire, and extract the electron-phonon coupling.}  We expect to see an anisotropic behavior in the phonon signal in RIXS spectrum, and afterwards we will carefully extract the electron-phonon coupling based on Ament model\cite{ament_determining_2011}. This is the starting point of dark matter detection.




\section{Planned Research Activities and Schedule}
\paragraph{Project 1: Charge and Lattice Dynamics in Cuprate Superconductors}
We've finished the first experimental phase of the first project, i.e. the RIXS measurement on $\mathrm{La_{1.8-x}Eu_{0.2}Sr_xCuO_{4}}$, with $x=0.125$, has been done earlier at beamline I21 at Diamond light source in the UK. More experiment might be in need due to the lack of data of $\mathrm{La_{1.8-x}Eu_{0.2}Sr_xCuO_{4}}$ with different doping level. We plan to apply for beamtime at I21 again in the next cycle.
\begin{itemize}
  \item \textit{Timeline}: Q1-Q2 2024, RIXS studies on cuprate superconductors.
  \item \textit{Methodology}: Analyzing phonon intensities at charge orders' wavevector using RIXS.
\end{itemize}

\paragraph{Project 2: Electron-Phonon Coupling in KTO and STO}
The proposal of $\mathrm{SrTiO_{3}}$ has been submitted to beamline I21 at Diamond light source, and was accepted ealier at the time of this application. The time of the beamtime is not determined yet, but will be most likely be scheduled in 2024. The proposal of $\mathrm{KTaO_{3}}$ has been submitted to beamline 41A at Taiwan Photon Source in Taiwan. 
\begin{itemize}
  \item \textit{Timeline}: Q3 2024-Q1 2025, RIXS experiments on LAO/KTO and LAO/STO interfaces.
  \item \textit{Methodology}: Quantitative comparison of electron-phonon coupling in different orientations.
\end{itemize}

\paragraph{Project 3: Sub-MeV Dark Matter Detection}
The initial experiment on sapphire has been done at beamline 41A at Taiwan Photon Source in Taiwan ealier at the time of this application. However, more experiment is needed due to the lack of high-resolution data in (001) direction of sapphire. We plan to apply for another beamtime at beamline ID32 at ESRF in France in the next cycle.
\begin{itemize}
  \item \textit{Timeline}: Q2-Q4 2025, RIXS studies on sapphire.
  \item \textit{Methodology}: Measuring electron-phonon coupling in sapphire to infer dark matter interactions.
\end{itemize}

\section{Available Resources}
All three projects require the use of synchrotron facilities, RIXS instruments. Access to synchrotron facilities and RIXS instruments hinges on successful approval of corresonding research proposals. The proposals are accepted in two cycles annually. 


Moreover, the machine learning project carried out in the group of Prof. Johan Chang at University of Zurich will greatly facilitate the experiment measurement and the data analysis afterwards. RIXS is now a crucial method for examining quantum materials, but its effectiveness is significantly hampered by the lengthy duration required for data acquisition and the relativly short allocation of beamtime. This machine learning project provides a way to denoise the RIXS data, and thus will be very helpful in accerlating the data acquisition process.  


\section{Significance of Expected Results}

The proposed research projects hold significant implications in the fields of condensed matter physics and dark matter detection. The comprehensive exploration of electron-phonon coupling in different systems through the advanced RIXS technique will certainly provide deeper insights across multiple domains.

\paragraph{Revealing the Nature of Charge Order Excitations in Cuprate Superconductors} While static charge order has been studied extensively in cuprate, dynamical charge order, or charge order exictations, is still not well explored due to the lack of experimental techiques before the advent of RIXS\cite{li_multiorbital_2020}. Thanks to the recent development in RIXS, it is now possible to study the charge order excitations in cuprate superconductors. Our research will provide a new perspective on the relation between phonon and charge order excitations in cuprate superconductors, and thus deepen our understanding of the interplay among phonon, charge order, and superconductivity. 

\paragraph{Advancing Understanding in High-Temperature Superconductivity}
The investigation of electron-phonon coupling in cuprate superconductors, $\mathrm{SrTiO_{3}}$ and $\mathrm{KTaO_{3}}$ is fundamental to unraveling the mysteries of high-temperature superconductivity. Current theories on unconventional superconductivity remain incomplete. Our research aims to bridge this knowledge gap, offering a deeper comprehension of the underlying mechanisms. We anticipate contributions to the undertanding of the role played by electron-phonon coupling in unconventional superconductivity, potentially leading to new theoretical frameworks. 

\paragraph{Contributions to Dark Matter Research}
Our investigation into the electron-phonon coupling in sapphire as a candidate for dark matter detection represents a pioneering effort in the search for dark matter. By carefully characterizing electron-phonon interaction, and potentially dark-matter-phonon couplings, this project could significantly impact the design and sensitivity of future phonon-based dark matter detection experiments. This is undoubtly the first step towards unmasking one of the most important problems in modern physics.

\paragraph{Methodological Advancements}
This research will also demonstrate the capabilities of RIXS as a tool for studying charg order and electron-phonon interactions. We not only aim to make use of the technique experimentally, but also computationally and threotically. We are looking to pushing the boundaries of this technique, developing new simulation method for more accurate extraction of electron-phonon coupling. This will certainly enhance the methodological framework available to the scientific community, potentially leading to new applications of RIXS in other areas of physics and materials science.

\paragraph{Providing data set for machine learning}
Due to the fact that the three projects are all based on RIXS measurements, it creates a playground and provides great opportunity for advancing the denosing technique based on machine learning developed in the group of Prof. Johan Chang at University of Zurich. The denosing technique will be very helpful in accelerating the data acquisition process, and thus will be promising in facilitating future RIXS experiments.



\bibliographystyle{unsrt}
\bibliography{ref}
\addcontentsline{toc}{chapter}{References}

\end{document}
